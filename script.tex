\documentclass[oneside,leqno,11pt]{amsart}
\usepackage[english]{babel}
\usepackage[utf8]{inputenc}

\usepackage[numbers, compress]{natbib}

\usepackage{Definitions}
\usepackage{xcolor}

\numberwithin{equation}{section}

\usepackage[a4paper,pdftex]{geometry}
\usepackage{fancyhdr}

\definecolor{orangegood}{RGB}{237,156,40}
\definecolor{bluegood}{RGB}{58,183,229}

\usepackage[pdftex,citecolor=bluegood,linkcolor=bluegood,urlcolor=bluegood,colorlinks=true]{hyperref}

\usepackage{titlesec}
\usepackage{color}
\usepackage{mathrsfs}

\titleformat{\section}{\sf\bfseries\Large\centering}{\thesection}{1em}{}
\titleformat{\subsection}{\sf\bfseries}{\thesubsection}{1em}{}
\titleformat{name=\paragraph,numberless}[runin]{\sf\slshape}{}{0em}{}[.]
\titlespacing{\paragraph}{0em}{0.5em}{0.5em}

\usepackage{bbm}
\newcommand{\indic}{\mathbbm{1}}
\newcommand{\norm}[1]{\|#1\|}

\linespread{1.2}

\begin{document}

\title{Optimal Transport}
\author{Alessio Figalli}
\date{Autumn 2019}

\maketitle

\section{Week 1}
\subsection{Introduction}
\paragraph{Monge, 1781} Have some soil extracted from earth and want to build a fortification. The cost is $c(x,y) = |x-y|$. The question is who goes where?

\paragraph{Kantorovic, 1940} Have some bakeries at $x_i$ and coffeeshops at $y_j$. Bakery $i$ produces $\alpha_i \geq 0$ amount of bread and coffee shop $j$ needs $\beta_j \geq 0$. Assume that $\sum \alpha_i = \sum \beta_j =  1$. Note that for Monge, the transport is ``deterministic''; $x$ is sent to $T(x)$. Kantorovic looked for matrices $(\gamma_{ij})_{i\in [N], j\in [M]}$ such that
\[
    \left\{ \begin{array}{l}
            \gamma_{ij} \geq 0, \\
            \alpha_i = \sum \gamma_{ij}, \forall i, \\
            \beta_j = \sum \gamma_{ij}, \forall j,\\
            \gamma_{ij} \text{ minimizes the cost } \sum_{i,j} \gamma_{ij}c(x_i, y_j).
    \end{array} \right.
\] 

\paragraph{Applications} $c(x,y) = |x-y|^2$: Euler equation, isoperimetric/Sobolev inequalities, and PDEs such as $\partial_t u = \Delta u$, $\partial_t u = \Delta(u^m)$, and $\partial_t u = \mathrm{div}(\nabla W \star_u u)$. 

For $c(x,y) = |x-y|^p$, there are cases for $p < 1$, $p=1$, and $p > 1$, the hardest one of which is $p=1$ (Monge tried to tackle the hardest!). It has applications to probability and kinetic theory.

Also one can have $c(x,y) = d^2(x,y)$ on $(M,g)$. It has applications to curvature. 

\begin{remark}
    All my spaces are metric spaces that are separable and complete. Most of the course is about $X = \RR^n$. Also, all measures  and maps are Borel.
\end{remark}

\subsection{A Rigorous Definition of Transport}
\begin{definition}[Image Measure]
    Given $X,Y$, let $T:X\to Y$. Let $\mu \in P(X)$, the set of probability measures on $X$. Define $T\#\mu \in P(Y)$ as follows:
    \[
        (T\#\mu)(A) = \mu(T^{-1}(A)).
    \] 
\end{definition}
\begin{lemma}
    $T\#\mu$ is a probability measure on $Y$.
\end{lemma}
\begin{proof}
    First, it is clear that $T\#\mu(\emptyset) = 0$ and $T\#\mu(Y) = 1$. Let $\{A_i\}_{i\in I}$ be a countable family of disjoint sets in $Y$. Then $\{T^{-1}(A_i)\}$ are disjoint in $X$. Also $\bigcup T^{-1}(A_i) = T^{-1}(\bigcup A_i)$. Taking measures from both sides results in additivity of $T\#\mu$.  
\end{proof}

\begin{remark}
    One may be tempted to do the following: to define a measure on $X$, \eg, set $(S^\#\nu)(E) = \nu(S(E))$. This is NOT a measure. Take the constant map and dirac measure as example. 
\end{remark}

\begin{lemma}\label{lem:testfun}
    Let $T:X\to Y$ and $\mu\in P(X)$. Then $\nu = T\#\mu$ iff $\forall \phi:Y\to\RR$ Borel and bounded, one has \[
        \int_Y \phi(y)\, d\nu(y) =  \int_X \phi(T(x))\,d\mu(x).
    \] 
\end{lemma}
\begin{corollary}
    \[
        \int_Y \phi\, d(T\#\mu) = \int_X \phi\circ T\, d\mu.
    \] 
\end{corollary}
\begin{proof}[Proof of Lemma~\ref{lem:testfun}]
    Real analysis induction: one can prove for all simple functions that the equality holds. It remains to prove that one can approximate any bounded function with simple functions. Let $\phi:Z\to \RR$ be Borel and bounded. Set $M\in\NN$ and define
    \[
        A_i := \{ \tfrac{i}{M} \leq \phi \leq \tfrac{i+1}{M} \}, \quad\forall i\in\ZZ,
    \] 
    and set $\phi_M = \sum_{i \in \ZZ} \frac{i}{M} \indic_{A_i}$. Note that $A_i = \emptyset$ for large $i$. Then 
    \[
        \norm{\phi - \phi_M}_{L^\infty} \leq \max_i \norm{\phi - \phi_M}_{L^\infty(A_i)} \leq \frac{1}{M}.
    \] 
    Hence, $\forall\sigma\in P(Z)$, \[
        \left| \int_Z (\phi - \phi_M)\, d\sigma\right| \leq \norm{\phi - \phi_M}_{L^\infty}\int_Z d\sigma \leq \frac{1}{M}.\qedhere
    \] 
\end{proof}
\begin{lemma}
    $(S \circ T)\# \mu = S\#(T\#\mu)$.
\end{lemma}
\begin{proof}
    For all $\phi$ one has
    \[
        \int \phi\, d(S\circ T)\#\mu = \int \phi \circ (S \circ T)\, d\mu = \int (\phi \circ S) \circ T\, d\mu = \int (\phi \circ S)\,dT\#\mu = \int \phi\, dS\#(T\#\mu). \qedhere
    \] 
\end{proof}

\subsection{Transport Maps}
Given $\mu \in P(X)$ and $\nu \in P(Y)$, $T:X\to Y$ is a \emph{transport map} from $\mu$ to $\nu$ if $T\#\mu = \nu$. 
\begin{remark}
    Given $\mu, \nu$ the set $\{T:T\#\mu = \nu\}$ can be empty! For example, take $\mu = \delta_{x_0}$ for $x_0 \in X$. Given any $T:X\to Y$, one has $T\#\mu = \delta_{T(x_0)}$ (Prove using Lemma~\ref{lem:testfun}).
\end{remark}

\begin{definition}[Coupling]
    $\gamma \in P(X\times Y)$ is a coupling of $\mu$ and $\nu$ if 
    \[
        (\Pi_X)\# \gamma = \mu, \quad (\Pi_Y)\# \gamma = \nu,
    \] 
    where $\Pi_X$ is projection on $X$. This condition is equivalent to having for all $\phi:X\to\RR$,
    \[
        \int_{X\times Y} \phi(x)\,d\gamma(x,y) = \int \phi \circ \Pi_X(x,y)\,d\gamma(x,y) = \int_X \phi(x)\,d\mu(x),
    \] 
    and for all $\psi:Y\to\RR$, $\int_{X\times Y} \psi(y)\,d\gamma(x,y) = \int_Y \psi(y)\,d\nu(y)$.
    We write $\gamma \in \Gamma(\mu, \nu)$ for a coupling.
\end{definition}
\begin{remark}
    Given $\mu, \nu$, the set $\{\gamma : \gamma\in \Gamma(\mu, \nu)\}$ is always nonempty. Indeed $\gamma = \mu\otimes\nu$ is a coupling. 
\end{remark}

\begin{remark}[Transport vs. Coupling]
    Let $T:X\to Y$ be such that $T\#\mu = \nu$. Set $(\mathrm{Id}\times T): X\to X\times Y$ be $x\mapsto (x, T(x))$. Define $\gamma_T = (\mathrm{Id}\times T)\# \mu \in P(X\times Y)$. We claim that $\gamma_T \in \Gamma(\mu, \nu)$. For proof, observe that
    \[
        (\Pi_X)\#\gamma_T = (\Pi_X)\#(\mathrm{Id}\times T)\#\mu = (\Pi_X \circ (\mathrm{Id}\times T))\# \mu = \mu,
    \] 
    and
    \[
        (\Pi_Y)\#\gamma_T = \cdots = \nu.
    \] 
\end{remark}

\subsection{Examples of Transport}
\paragraph{Measurable Transport}
One has the following theorem:
\begin{theorem}
    Let $\mu \in P(X)$ such that $\mu$ has no atoms. Then there exists $T:X\to\RR$ such that $T$ is injective $\mu$-a.e. and $T\#\mu = dx|_{[0,1]}$ and $T^{-1}:[0,1]\to X$ exists a.e. with $(T^{-1})\#dx = \mu$.
\end{theorem}

\paragraph{Monotone Arrangement} Assume $\mu, \nu$ be measures on $\RR$. Let $F(x) := \int_{-\infty}^x d\mu(t)$ and $G(y) := \int_{-\infty}^y d\nu(t)$. By convention, we assume that $G(y) = \lim_{\epsilon \to 0+} G(y-\epsilon)$. Define $G^{-1}(t) := \inf \{y: G(y) > t\}$ and let $T := G^{-1}\circ F:\RR\to \RR$.

\begin{theorem}
    If $\mu$ has no atoms, then $T\#\mu = \nu$.
\end{theorem}
\begin{proof}
    It suffices to show that $\nu(A) = \mu(T^{-1}(A))$ for all $A=(-\infty, a)$. 
\end{proof}

\end{document}
